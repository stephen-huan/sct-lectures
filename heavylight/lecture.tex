\documentclass[11pt, oneside]{article}
\usepackage{titling, hyperref, geometry, amsmath, amssymb, algorithm, graphicx, textcomp, subcaption, cancel}
\usepackage[noend]{algpseudocode}
\usepackage[cache=false]{minted}
\geometry{a4paper}

\hypersetup{
    colorlinks=true,
    urlcolor=cyan
}

\newcommand{\emphasis}[1]{\textcolor{blue}{\textbf{\textit{#1}}}}

\title{Heavy-Light Decomposition}
\author{Stephen Huan}

\begin{document}
\maketitle

\section{Sample Problems}

\begin{enumerate}
  % \item \href{http://www.spoj.com/problems/QTREE/}{SPOJ QTREE}
  % \item \href{http://www.spoj.com/problems/QTREE2/}{SPOJ QTREE2}
  % \item \href{http://www.spoj.com/problems/QTREE3/}{SPOJ QTREE3}
  % \item \href{http://www.spoj.com/problems/QTREE4/}{SPOJ QTREE4}
  % \item \href{http://www.spoj.com/problems/QTREE5/}{SPOJ QTREE5}
  % \item \href{http://www.spoj.com/problems/QTREE6/}{SPOJ QTREE6}
  % \item \href{http://www.spoj.com/problems/QTREE7/}{SPOJ QTREE7}
  % \item \href{http://www.spoj.com/problems/COT/}{SPOJ COT}
  % \item \href{http://www.spoj.com/problems/COT2/}{SPOJ COT2}
  % \item \href{http://www.spoj.com/problems/COT3/}{SPOJ COT3}
  % \item \href{http://www.spoj.com/problems/GOT/}{SPOJ GOT}
  % \item \href{http://www.spoj.com/problems/GRASSPLA/}{SPOJ GRASSPLA}
  % \item \href{http://www.spoj.com/problems/GSS7/}{SPOJ GSS7}
  % \item \href{http://www.codechef.com/problems/GERALD2}{CODECHEF GERALD2}
  % \item \href{http://www.codechef.com/problems/RRTREE}{CODECHEF RRTREE}
  % \item \href{http://www.codechef.com/problems/QUERY}{CODECHEF QUERY}
  % \item \href{http://www.codechef.com/problems/QTREE}{CODECHEF QTREE}
  % \item \href{http://www.codechef.com/problems/DGCD}{CODECHEF DGCD}
  % \item \href{http://www.codechef.com/problems/MONOPLOY}{CODECHEF MONOPLOY}
  \item \href{http://usaco.org/index.php?page=viewproblem2&cpid=970}{USACO Milk Visits} - Farmer John has a tree with \( N \) (\(1 \leq N \leq 10^5 \)) nodes.
  Each node has a type of milk. You are asked \( M \) (\(1 \leq M \leq 10^5 \)) queries where each query gives a start node \( A \), end node \( B \), and type of milk \( C \).
  Answer whether the type of milk \( C \) is found on the path from \( A \) to \( B \) for each query.

  Solution: Apply HLD. Then, the only problem is finding what data structure to associate with each path.
  First, we'll associate a data structure with a heavy path by associating heavy paths by their head
  as discussed above. Then, for each node on a heavy path add its type to an dictionary
  mapping milk type to occurrences, sorted by depth. When walking up the tree to answer queries,
  compute whether a heavy path will contain a particular milk type by binary searching on the
  current depth of the node that's going to be raised in the tree in the corresponding occurence list.
  Finally, when both nodes are on the same heavy path, test whether a milk type is between them
  by binary searching on the left for the higher node, on the right for the lower node,
  and seeing whether the two indexes are different.

  \item \href{http://www.usaco.org/index.php?page=viewproblem2&cpid=102}{USACO Grass Planting}
  \item \href{http://www.usaco.org/index.php?page=viewproblem2&cpid=921}{USACO Cowland}

\end{enumerate}

\section{Past Lectures}

\begin{enumerate}
  \item \href{}{``Heavy Light Decomposition'' (Pranav Mathur, 2020)}
  \item \href{https://activities.tjhsst.edu/sct/lectures/1819/2019_3_15_HLD.pdf}{``Heavy Light Decomposition'' [Identical] (Daniel Wisdom, 2019)}
  \item \href{https://activities.tjhsst.edu/sct/lectures/1718/2018-03-23_Heavy_Light_Decomposition.pdf}{``Heavy Light Decomposition'' (Daniel Wisdom, 2018)}
  \item \href{https://activities.tjhsst.edu/sct/lectures/1415/SCT_Heavy-Light_Decomposition.pdf}{``Heavy-Light Decomposition'' (Samuel Hsiang, 2015)}
  \item (Unavailable) ``Heavy-Light Decomposition'' (Wassim Omais, 2017)
\end{enumerate}

\section{Works Cited}

\begin{enumerate}
  \item \href{https://blog.anudeep2011.com/heavy-light-decomposition/}{Sample Problems}
  \item \href{www.csie.ntnu.edu.tw/ u91029/Heavy-LightDecomposition1.png}{Image}
\end{enumerate}


\end{document}
