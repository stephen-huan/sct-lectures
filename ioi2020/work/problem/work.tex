\documentclass[11pt, oneside]{article}
\usepackage{titling, hyperref, geometry, amsmath, amssymb, algorithm, graphicx, textcomp, subcaption, cancel, float}
\usepackage[noend]{algpseudocode}
\usepackage[cache=false]{minted}
\geometry{a4paper}

\hypersetup{
  colorlinks=true,
  urlcolor=cyan
}

\newcommand{\emphasis}[1]{\textcolor{blue}{\textbf{\textit{#1}}}}
\newcommand{\smallemphasis}[1]{\textbf{\textit{#1}}}

\begin{document}
\section{Problem}

Bessie is using her supercow strength to push a tractor around for Farmer John,
but because she is health-conscious, she wants to measure how much energy
she exerts in order to plan out her daily caloric intake.
She starts off in a field \( N \) (\(1 \leq N \leq 1000\)) by \( M \) (\( 1 \leq M \leq 1000\))
and has to push the tractor along certain paths,
at most \( Q \) (\( 1 \leq Q \leq 1000\))
which can intersect themselves and are contained in the grid.
Given the force she exerts at every point,
calculate the total work done over all the paths. \\
Recall that \( W = \vec{F} \cdot \vec{d} \)
where \( W \) is the work, \( \vec{F} \) is the force, and \( \vec{d} \) is the displacement. \\

\noindent
A path is described by an integer starting point \( (a, b) \) where \( 0 \leq a \leq N \) and \(0 \leq b \leq M \),
and then followed by at most \( 10^3 \) ``tokens''.
Each token specifies one of the 4 cardinal directions
and an integer length to move in that direction for.
Note that the force can change through a token which changes the work along a token. \\

\noindent
\textbf{INPUT FORMAT:} \\
The first line contains \( N \), \( M \), and \( Q \). \\
Each line in the next \( N \) lines contains \( M \) pairs of integers
which correspond to the \\ horizontal and vertical components of the force at that point. \\
The next \( Q \) lines contain each path, as described above. \\

\noindent
\textbf{OUTPUT FORMAT:} \\
A single integer representing the total work Bessie does over all the paths. \\

\noindent
\textbf{SAMPLE INPUT:} \\
4 4 3 \\
1 4 -2 3 3 -2 4 -2 \\
-3 2 3 2 1 0 2 3 \\
0 3 -4 2 5 1 -3 2 \\
1 2 2 -3 2 3 -1 5 \\
0 0 1 E 2 S 1 E \\
0 2 1 S 1 W 1 N 1 E \\
1 3 1 N 1 S 1 W 1 E  \\

\noindent
\textbf{SAMPLE OUTPUT:} \\
-11

\newpage

\begin{minted}{text}
The force grid looks like this:
(1, 4) (-2, 3) (3, -2) (4, -2)
(-3, 2) (3, 2) (1, 0) (2, 3)
(0, 3) (-4, 2) (5, 1) (-3, 2)
(1, 2) (2, -3) (2, 3) (-1, 5)\end{minted}

\noindent
On the first path, Bessie starts out at 0, 0, then moves one east (1*1),
2 south (3*1 + 2*1) then one east (-4*1) for a total of 2 work. \\

\noindent
On the second path, Bessie starts out at 0, 2,
then moves one south (-2*1),
one west (1*-1), one north (2*-1), one east (-2*1) for a total of -7 work. \\

\noindent
On the third path, Bessie starts out at 1, 3,
then moves one north (3*-1), one south (-2*1),
one west (2*-1) and one east (1*1) for a total of -6 work. \\

\( 2 - 7 - 6 = -11 \)

\newpage

\section{Solutions}
\subsection{Naive}

Simply simulate each query, walking along the path and computing the dot product every time.
This runs in \( O(QTN) \) in the worst case, where \( T \) is the number of tokens, which is too slow for the given bounds.

\begin{minted}{python}
N, M, Q = map(int, input().split())

# force matricies
forcex, forcey = [[0]*M for i in range(N)], [[0]*M for i in range(N)]
for i in range(N):
    line = list(map(int, input().split()))
    for j in range(0, 2*M, 2):
        forcex[i][j//2] = line[j]
        forcey[i][j//2] = line[j + 1]

# direction lookup
dirs = {"N": (-1, 0), "E": (0, 1), "S": (1, 0), "W": (0, -1)}
work = 0

for i in range(Q):
    line = input().split()
    a, b, line = int(line[0]), int(line[1]), line[2:]
    path = 0
    for j in range(0, len(line), 2):
        # vector in magnitude-direction notation
        mag, dir = int(line[j]), dirs[line[j + 1]]
        for k in range(mag):
            # dot product
            path += dir[0]*forcey[a][b] + dir[1]*forcex[a][b]
            # vector addition
            a, b = a + dir[0], b + dir[1]
    work += path

print(work)
\end{minted}

\newpage

\subsection{Full Credit}

The solution relies on the fact that in the dot product, the displacement vector
doesn't change and thus can be factored out. Also, the x and y directions
can be treated independently. Thus, the answer is just the prefix sum down
the particular row or column traversed by the displacement vector.

\begin{minted}{python}
N, M, Q = map(int, input().split())

# force matricies
forcex, forcey = [[0]*M for i in range(N)], [[0]*M for i in range(N)]
for i in range(N):
    line = list(map(int, input().split()))
    for j in range(0, 2*M, 2):
        forcex[i][j//2] = line[j]
        forcey[i][j//2] = line[j + 1]

# prefix sums in x and y directions
def query(prefix, x1, y1, x2, y2):
    return prefix[x2 + 1][y2 + 1] - prefix[x2 + 1][y1] \
         - prefix[x1][y2 + 1] + prefix[x1][y1]

prefixx = [[0]*(M + 1) for i in range(N + 1)]
prefixy = [[0]*(M + 1) for i in range(N + 1)]

for i in range(1, len(prefixx)):
    for j in range(1, len(prefixx[0])):
        prefixx[i][j] += prefixx[i - 1][j] + forcex[i - 1][j - 1]

for i in range(1, len(prefixx)):
    for j in range(1, len(prefixx[0])):
        prefixx[i][j] += prefixx[i][j - 1]

for i in range(1, len(prefixy)):
    for j in range(len(prefixy[0])):
        prefixy[i][j] += prefixy[i - 1][j] + forcey[i - 1][j - 1]

for i in range(1, len(prefixy)):
    for j in range(1, len(prefixy[0])):
        prefixy[i][j] += prefixy[i][j - 1]

# direction lookup
dirs = {"N": (-1, 0), "E": (0, 1), "S": (1, 0), "W": (0, -1)}
work = 0

for i in range(Q):
    line = input().split()
    a, b, line = int(line[0]), int(line[1]), line[2:]
    path = 0
    for j in range(0, len(line), 2):
        # vector in magnitude-direction notation
        mag, dir = int(line[j]), dirs[line[j + 1]]
        # use prefix sums to compute dot product over the entire vector
        mag -= 1
        path += query(prefixy, a, b, a + mag*dir[0], b + mag*dir[1])*dir[0] + \
                query(prefixx, a, b, a + mag*dir[0], b + mag*dir[1])*dir[1]
        # vector addition
        mag += 1
        a, b = a + mag*dir[0], b + mag*dir[1]
    work += path

print(work)
\end{minted}

This runs in \( O(QT) \), which is sufficient.

\newpage

\subsection{Misc}

I orginally planned this problem as a way to build up a ``discrete gradient'',
in the same way that a prefix sum is a discrete integral.
Unfortunately, the fundamental theorem of line integrals no longer works
on the discrete case, or at least I couldn't think of a way to get it to work.

My code is available here:
\begin{minted}{python}
# precompute the gradient to the vector field, such that
# grad f = F = <x, 2y> (where grad f = <df/dx, df/dy>)
# we know the vector field is conservative so the gradient exists
grad = [[0]*M for i in range(N)]
for y in range(1, N):
    for x in range(1, M):
        # propagate differential in the x direction
        grad[y][x] = x - 1 + grad[y][x - 1]
    if y < N - 1:
        # propagate differential in the y direction
        grad[y + 1][0] = 2*y + grad[y][0]

for i in range(Q):
    line = input().split()
    a, b, line = int(line[0]), int(line[1]), line[2:]
    # endpoint
    c, d = a, b
    for j in range(0, len(line), 2):
        # vector in magnitude-direction notation
        mag, dir = int(line[j]), dirs[line[j + 1]]
        # vector addition
        c, d = c + mag*dir[0], d + mag*dir[1]
    # line integral^b_a of F dr = f(r(b)) - f(r(a))
    print(grad[c][d] - grad[a][b])
    work += grad[c][d] - grad[a][b]
\end{minted}

\end{document}
